\documentclass[11pt, a4paper, titlepage]{article}
\usepackage[margin=3.5cm]{geometry}
\usepackage{graphicx}
\usepackage{xcolor}
\usepackage[utf8]{inputenc}
\usepackage[english]{babel}
\usepackage[style=ieee, block=ragged]{biblatex}



\begin{document}
	\begin{titlepage}
		% Logo at the top within the margins
		\noindent
		\includegraphics[width=0.15\paperwidth]{logo-mg.png} % Adjust the size as needed
		
		% Title and subtitle with left offset
		\vspace*{0.5\paperwidth} % Adjust the vertical space as needed
		\noindent\hspace*{0.15\paperwidth} % Offset from the left
		\begin{minipage}{\textwidth}
		
			\color[HTML]{0b5394}\Huge Gymnasiearbetets namn \\
			\color{black}\Large En underrubrik om man vill \\
		\end{minipage}
	
		\vfill % Push the authorship information to the bottom of the page
		\noindent
		\begin{minipage}{\textwidth}
			\normalsize % 12pt font size
			Förnamn och efternamn \\
			\textbf{Gymnasiearbete 100 poäng} \\
			\textbf{Klass:} ? TE \\
			\textit{Samhällsvetenskapsprogrammet} \\
			\textbf{Läsåret:} 20/20? \\
			\textbf{Handledare:}
		\end{minipage}
	\end{titlepage}
	\newpage \ \newpage
	\begin{abstract}
		Abstract (sammanfattning på svenska) skrivs sist, när gymnasiearbetet är klart. Texten skrivs på engelska och består av en presentation av ämnet, en kort beskrivning av din undersökning, en ännu kortare beskrivning av metod och material, samt vilken slutsats du har kommit fram till, dvs. ditt resultat. Texten i ett abstract kan liknas vid en baksidestext på en bok, där läsaren snabbt ska få veta vad arbetet handlar om. Skillnaden är att du avslöjar resultatet och gör en s.k. “spoiler”. Abstractet ska hållas kort och koncist, cirka 150-300 ord, och skrivas med löpande text, dvs. inga punkter, bilder, diagram, figurer, eller liknande. En läsare ska alltså få en god helhetsbild av ditt gymnasiearbete bara genom att läsa ditt abstract.
		\newline\newline
		Se Gymnasiearbetet - en handbok (Andersson \& Etzler, 2017) s. 126 för mer information. 
		
		\begin{flushleft}
			{\small {\bf Keywords:} Några, Användbara, Nyckelord}
		\end{flushleft}
	\end{abstract}

\refstepcounter{page}
\setcounter{page}{2}
\tableofcontents
\newpage

\section{Inledning}
Här berättar du kort och intresseväckande om ditt ämne, varför du har valt det och varför det är intressant (för dig). Känn dig fri att vara personlig! Tänk inledningen som en säljpitch. Redogör gärna för bakgrundsfakta här, om det finns.  

\textbf{Exempel:}
Glaciärerna smälter, folk drunknar i Medelhavet, Nordkorea skickar missiler och människor tigger på gatorna. Allt det där är politikens roll att lösa. Men vem tror på politiken längre? Inte jag! Min generation (ungdomar födda efter år 2000) styrs av media. Själv är jag förstagångsväljare och har ingen aning om vilka jag ska rösta på. Alla säger ju samma sak. Samtidigt går X-partiet starkt fram och får många väljare (etc, etc, etc)... 

\subsection{Syfte och frågeställningar}

\begin{itemize}
	\item Syftet = Vad ska du undersöka och varför?
	\item Frågeställning = Vilken/vilka huvudfrågor har du? OBS! Inga ja-/nej-frågor!
\end{itemize}
Formulera syftet kort och tydligt, cirka tre meningar. Skriv inget personligt, som i inledningen. Skriv gärna frågeställningen/-arna i punktform. Hela ditt gymnasiearbete går ut på att besvara frågeställningen/-arna. Skriv därför frågeställning/-ar som hjälper dig själv framåt! 

\textbf{Exempel:}
\begin{itemize}
	\item Hur har X-partiets frågor presenterats i medierna under valrörelsen (årtal)? 
	\item Vilka av X-partiets frågor har synts mest i medierna och hur tolkas det av unga och nya väljare? 
\end{itemize}


\subsection{Tidigare forskning}
Här redovisar du vad andra redan har skrivit inom området som berör ditt syfte och frågeställning. Du redovisar detta genom referat och citat. Tänk på att redovisa dina källor i löptext. 
\subsection{Material och metod}
Det här avsnittet skrivs i dåtid. Du beskriver kort och gått hur du har gått tillväga för att besvara dina frågeställningar. Vilket material och metod har du använt dig av? Det vetenskapliga syftet med avsnittet är att någon annan ska kunna upprepa ditt arbete och komma fram till samma (eller en annan) slutsats, precis som när man lagar en maträtt och följer ett recept. Här finns ingen plats för åsikter, funderingar eller personliga input. Om du genomfört intervjuer och enkäter – beskriv också vilket urval och vilka avgränsningar du gjort. Hur valde du vilka personer som fick delta i intervjun? Vilka fick ta del av enkäten? Om du endast kommer att intervjua klasskamrater – skriv det! Lägg också till ”för att”, så att du motiverar varför. Beskriv också kort varför du valt att använda de material och metoder du valt. 

\textbf{Exempel:} 
Metoden som använts är att intervjua förstagångsväljare som får frågor om mediebevakningen av X-partiet... 

\textbf{Exempel:} 
Intervjuerna som genomförts bestod av slumpmässigt utvalda personer mitt i stan, för att få en större bredd av intervjupersoner. Urvalet bestod av ungdomar som är förstagångsväljare.
\newline\newline
För fler exempel se sidan 117-118 i Gymnasiearbetet - en handbok (Andersson \& Etzler, 2017).

\subsection{Avgränsning}

\newpage
\section{Resultat}
Här presenterar du resultatet av din undersökning. Spalta upp i rubriker och underrubriker så att det blir tydligt för dig själv och för din läsare. I denna del redovisar samt "svarar" du på uppsatsens syfte och frågeställningar. Referenser måste finnas löpande i texten, se; hänvisningar i text. OBS! Inga egna åsikter. Dina egna analyser och åsikter får du lyfta fram i diskussionsdelen.

\textbf{Exempel:} 
I filmen Intervjuteknik (2014) visar filmaren och psykologen Bengt Bengtsson att ett bra sätt att få personer att svara på frågor är att ställa öppna frågor. Med öppna frågor menar Bengtsson frågor som inte har ett ja- eller nej-svar. (etc...). Men i boken Ställ en fråga menar Anna Andersson att... (etc)... 
\newline\newline
De filmade intervjuerna genomfördes mitt på Sergels torg i Stockholm under lunchtid. Antalet ungdomar som intervjuades blev 20. Fem visade sig för unga för att få rösta i valet 20XX. Alltså var det bara 15 ungdomar som passade in i målgruppen. Av dessa 15 hade bara tio följt X-partiets frågor i media. (osv, osv, osv...).


\newpage
\section{Slutsatser/Samanfattning}
I denna del ska du börja med att kort sammanfatta uppsatsen. Börja med att lyfta fram uppsatsens syfte och frågeställningar samt hur du gick tillväga (arbetsprocessen). Därefter presenterar du slutsatserna, vad kom uppsatsen fram till (inte dina egna åsikter). Du presenterar slutsatserna utifrån vad som kommit fram i resultatdelen. Detta kan vidare jämföras med det som presenterats i tidigare forskning. OBS! I denna del får du inte ta med ny fakta och information. 
\newline\newline
Beskriv därefter vad ditt resultat lett fram till.
Exempelvis:  “Intervjuerna visade att de flesta ansåg att …” eller “Min jämförelse visar på…”. 

\newpage
\section{Avslutande diskussion}
I denna del får du dra egna slutsatser av ditt resultat. Du kan här, precis som i inledningen ha en mer personlig ton. Tänk på att det tydligt ska framgå när det är dina egna åsikter och tolkningar. Det gör du genom formuleringar som “jag anser” eller “min åsikt är…”. Avsnittet kan avslutas med framåtsyftande frågor och dina tankar om hur man fortsatt kan studera det här ämnet. 


\newpage
\section{Källförteckning}
\newpage
\section{Bilagor}
Bilagor kan vara exempelvis enkäter, intervjufrågor, statistik, foton, print screen-bilder, e-mail-konversationer, etc. De tar bara onödig plats i själva rapporten, men är viktiga bakgrundsfakta som du kan hänvisa till i texten inom parentes (som en vanlig källa). Då skriver du (se Bilaga 1) när du pratar om din bilaga, eller infogar en fotnot. Du kan alltså ha fler bilagor och döpa dem till Bilaga 1, Bilaga 2, etc. Du kan också lägga till namnet på det som bilagan innehåller, tex. “Bilaga 1 - hemsida X”. 

	
\end{document}
