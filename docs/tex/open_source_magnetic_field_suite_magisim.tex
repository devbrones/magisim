\documentclass[]{report}


% Title Page
\title{Open source magnetic field simulations using parallelized computing on consumer hardware.}
\author{Jonas Cronholm, Rasmus C. Ljung}


\begin{document}
\maketitle

\begin{abstract}
	
	% THIS IS ONLY FOR BASE STRUCTURING!!! 
	% NEEDS A FULL REWRITE IN ORDER TO BE COMPLIANT WITH RULESET !
	
	With serious competition in the modern market it is not trivial to create a product that can match the expectations of professional users and still appeal to hobby engineers, especially at a modest price point. The gap in capabilities between enterprise and open source simulation software is growing bigger due to lack of funding, which often results in limited research and development for open source projects. This growing disparity poses a significant challenge for developers aiming to balance advanced features with affordability. Striking a harmony between meeting the rigorous demands of professionals while remaining accessible to enthusiasts is a complex task.
	
	Enterprise simulation software benefits from substantial financial backing, enabling teams to invest in cutting-edge research, continuous enhancements, and dedicated customer support. This translates to robust features, high accuracy, and comprehensive technical support, making them the preferred choice for intricate and mission-critical projects. However, the cost associated with these solutions can be prohibitive for smaller businesses and individual hobbyists.
	
	This gap is further exacerbated by the intricate nature of simulation software. Meeting the needs of professionals frequently involves complex algorithms, intricate modelling, and high-performance computing. Striving to offer similar capabilities within a modest price range for hobbyists can be daunting, as the associated costs can quickly spiral upwards.
	
	This paper presents a proof-of-concept software project that aims to illustrate the potential of open source magnetic field simulations utilizing parallelized computing techniques on consumer-grade hardware. Magnetic field simulations hold immense significance across scientific, engineering, and technological domains, yet their resource-intensive nature often limits accessibility, particularly in the context of proprietary software and high-cost hardware.
	
	By conducting a thorough analysis of existing open source simulation tools, parallel computing frameworks, and optimization strategies, we lay the groundwork for a proof-of-concept that showcases the potential of democratizing magnetic field simulations. This paper provides an in-depth exposition of the technical underpinnings involved in parallelization implementation, optimization techniques, and comprehensive performance evaluations. Through our preliminary experimentation, we offer empirical evidence of substantial reductions in simulation runtime while maintaining accuracy levels comparable to established proprietary solutions.
	
	Moreover, the paper acknowledges and addresses the inherent challenges intrinsic to parallel programming, hardware limitations, and scalability concerns when deploying simulations on consumer-grade hardware. By providing practical insights and strategic recommendations, we outline potential avenues to navigate these challenges and attain optimal performance across varying hardware configurations.
	
\end{abstract}

\section{Computation}
A regular computer usually performs tasks serially, one operation at a time. While this might be the most efficient method for everyday tasks, it does not apply to computationally heavy tasks spanning multiple dimensions.
One example of such a 



$n-1$ dimensional computations are possible using a graphics card due to the ability to parallelise computational tasks. 

$t(d,)$ 

\end{document}          
